
%%% Le chargement des packages %%%


\usepackage{amssymb}
\setlength{\columnsep}{1cm}
\usepackage{multicol}



\usepackage{eurosym}
\usepackage{graphics}
\usepackage{amsmath}
\usepackage{amsthm}  % DOIT être chargé AVANT de définir les théorèmes
% For Landau's notation
\usepackage{mismath}

\usepackage{stmaryrd} % Where \llbracket and \rrbracket are defined

\usepackage[french]{babel}
\usepackage[utf8]{inputenc}  
% \usepackage[latin1]{inputenc}
\usepackage[T1]{fontenc}  
\usepackage{pstricks-add}
\usepackage{graphicx}
\usepackage[T1]{fontenc}
\usepackage{babel,varioref,float} % polices
\usepackage{pgf,tikz} % permet de créer comme pstricks des figures en code LateX
\let\clipbox\relax

\usetikzlibrary{calc,decorations.pathmorphing,arrows}


\usepackage{pgflibraryarrows} % librairie liée à tikz
\usepackage{pgflibrarysnakes}
\usepackage{xcolor} % module de couleur pour tikz
\usepackage{pst-eucl}
\usepackage{titlesec} % Personnaliser les titres
\usepackage{graphicx}


%\theoremstyle{plain}
\newtheorem*{thm}{Théorème}
\newtheorem*{de}{Définition}
\newtheorem*{lem}{Lemme}


\usepackage{amsmath,amsfonts,graphicx,geometry}
\geometry{top=1.9cm,left=1.2cm,right=1.2cm,bottom=1.6cm}
\usepackage{multicol}



\usepackage{enumitem}

\usepackage{calc,color}
\usepackage{fancybox}
\usepackage{lastpage}
\usepackage{fancyhdr}
\usepackage{lipsum}

\usepackage{marginnote}
\usepackage{fontawesome}

\usepackage[weather]{ifsym}
\usepackage{pifont}


%\usepackage[utf8]{inputenc}    % Pour utiliser les lettres accentuées
%\usepackage[francais]{babel}    % Pour la langue française
\usepackage{enumitem} % Itemize, enumerate etc.
%\usepackage{enumerate} % Idem, mais a ne pas activer ensemble
\usepackage{amscd,amsmath} % Les packages de Maths
\usepackage{color} % les images
\usepackage{index} % Pour pouvoir créer un index
%\usepackage{titlesec} % Personnaliser les titres
\usepackage{mathdots} % Faire des petits points verticaux, en oblique etc
\usepackage{import} % Pouvoir importer des fichers dans un autre dossier
\usepackage[all]{xy} % Faire des zoulis diagrammes
 \usepackage[cyr]{aeguill} % Guillemets en francais et autres
\usepackage{latexsym,stmaryrd,xypic,mathrsfs} % Différents package pour faire des maths
%%% ------------------------------------------ %%%
\usepackage{pgf,tikz} % permet de créer comme pstricks des figures en code LateX



\usepackage[most]{tcolorbox}
\usepackage{adjustbox}

%\usepackage{lmodern} % Police moderne
\usepackage{xcolor}


% Style du bloc (SANS ombre)
\tcbset{
  developpementsbox/.style={
    enhanced,
    sharp corners=south,  % coins carrés en bas
    colback=white,
    colframe=gold,
   % boxrule=1.2pt,
    borderline west={3mm}{0pt}{gold}, % barre épaisse à gauche
    left=10pt,
    right=8pt,
    top=6pt,
    bottom=6pt,
    before skip=12pt,
    after skip=12pt,
  }
}

% Style des items : numérotation classique
\newlist{devlist}{enumerate}{1}
\setlist[devlist,1]{%
  label=\textbf{\arabic*.},
  left=0pt, labelsep=1em, itemsep=0.7em,
  font=\normalsize
}

% Environnement avec le titre toujours affiché
\newenvironment{developpements}{%
  \begin{tcolorbox}[developpementsbox]
  \textbf{\large Liste des développements :}
  \vspace{4pt}
  \begin{devlist}
}{%
  \end{devlist}
  \end{tcolorbox}
}






\newcommand{\poubelle}[1]{} % Commande Poubelle : "\poubelle{ blabla}" et le blabla sera ignoré, même si il tient en 200 lignes





\definecolor{albertBlue1}{RGB}{34, 20, 81}
\definecolor{albertBlue2}{RGB}{54, 37, 157}
\definecolor{albertBlue3}{RGB}{173, 218, 247}

\definecolor{inter}{RGB}{156, 66, 167}

%%% Les couleurs %%%
\definecolor{fondtitre}{rgb}{0.20,0.43,0.09}  % vert fonce
\definecolor{gold}{RGB}{223,179,85}  % Gold
%\definecolor{gold2}{RGB}{255,214,100}  % Gold
\definecolor{gold2}{RGB}{53,93,173}   % Gold2 %bleu
\definecolor{gold3}{RGB}{255,214,100} 
\definecolor{goldinv}{RGB}{32,76,170}
\definecolor{bleu}{RGB}{53,93,173}  
\definecolor{vert}{RGB}{155,205,50}  
\definecolor{vertfoncé}{RGB}{0,128,32}  
\definecolor{bleufoncé}{RGB}{0,32,128}  
\definecolor{grenat}{RGB}{110,11,20}  
\definecolor{amande}{RGB}{130,196,108} 
\definecolor{magentapink}{RGB}{204,51,139}   
\definecolor{glauque}{RGB}{100,155,136}   
\definecolor{oeuf}{RGB}{239,213,178}   
\definecolor{bisque}{RGB}{255,228,196}   
\definecolor{inter}{RGB}{156, 66, 167}
%\definecolor{backg}{RGB}{185,220,200}   
%\definecolor{backg}{RGB}{215,224,240}
\definecolor{crème}{RGB}{253,247,196}  
\definecolor{noirleger}{RGB}{79,77,81}  
\definecolor{marron}{RGB}{88,41,0} 
\definecolor{gris}{RGB}{150,150,150} 
\definecolor{rougem}{RGB}{200,0,0} 
\definecolor{vertm}{RGB}{0,200,0} 
\definecolor{bleum}{RGB}{0,0,200} 
\definecolor{backg}{RGB}{252, 233, 240}





\setcounter{secnumdepth}{1} %  Profondeur des Chapitre/Sections/Sous-sections Etc

\setcounter{tocdepth}{2}  % Profondeur de la table principale



\titleformat{\chapter}[display]
{\clearpage\normalfont\centering\color{gold3}}
{\vspace*{\fill}\vspace{-5em}\fontsize{50}{48}\selectfont\bfseries\chaptertitlename\space\thechapter}
{20pt}
{\fontsize{40}{48}\selectfont\bfseries\thispagestyle{empty}}
[\vspace*{\fill}]
\titlespacing*{\chapter}
{0pt}     
{0pt}      
{0pt}




%\titleformat{\section}
%{\normalfont\Large\bfseries\color{gold2}}  % Format du titre
%{\thesection}   % Le numéro de section
%{1em}          % Espace entre le numéro et le titre
%{}     


%\makeatletter 
%\renewcommand\thesection{\@arabic\c@section} 
%\makeatothe

%\usepackage[export]{adjustbox}


%\renewcommand{\thechapter}{\Roman{chapter}}





% Properties
\definecolor{propColor}{RGB}{0, 255, 0}
\newcommand{\Props}[2][]{
	\begin{tcolorbox}[colback=propColor!5!white,colframe=propColor!75!black,title=Propositions #1]
		#2
\end{tcolorbox}}
\newcommand{\Prop}[2][]{
	\begin{tcolorbox}[colback=vertfoncé!5!white,colframe=vertfoncé!75!black,title=Proposition #1]
		#2
\end{tcolorbox}}

% Definitions
\definecolor{defColor}{RGB}{255, 255, 255}
\newcommand{\Defs}[2][]{
	\begin{tcolorbox}[colback=defColor!5!white,colframe=defColor!75!black,title=Definitions #1]
		#2
\end{tcolorbox}}
\newcommand{\Def}[2][]{
	\begin{tcolorbox}[colback=defColor!5!white,colframe=defColor!75!black,title=Definition #1]
		#2
\end{tcolorbox}}

% Theorems
\definecolor{theoremColor}{RGB}{255, 0, 0}
\newcommand{\Thm}[2][]{
	\begin{tcolorbox}[colback=bleufoncé!5!white,colframe=bleufoncé!75!black,title=Théorème #1]
		#2
\end{tcolorbox}}

% Applications
\newcommand{\Ap}[2][]{
	\begin{tcolorbox}[colback=grenat!5!white,colframe=grenat!75!black,title=Application #1]
		#2
\end{tcolorbox}}

% Examples
\newcommand{\example}[0]{
	\vspace{1em}
	{\bf Example :}
}

% Diverse things to highlight
\newcommand{\highlight}[1]{
	\vspace{1em}
	{\bf #1}
	\vspace{0.25em}
	
}



%\newtheorem{prop}{Proposition}
%\newtheorem{lem}{Lemme}
%\newtheorem{cor}{Corollaire}

\poubelle{
\newtheorem{thm}{Théorème}[]
\newtheorem{de}{Définition}[]
\newtheorem{prop}{Proposition}[]
\newtheorem{ap}{Application}[]
\newtheorem{rem}{Remarque}[]
\newtheorem{cex}{Contre-Exemple}[]
\newtheorem{cor}{Corollaire}[]
\newtheorem{cocor}{Cocorollaire}[]
\newtheorem{ex}{Exemple}[]
\newtheorem{lem}{Lemme}[]
\newtheorem{de-prop}{Définition-Proposition}[]
%\theoremstyle{definition} % à partir de là, les newtheorem seront en lettre droite contrairement à italique.
\newtheorem{etape}{Etape}[]
\newtheorem{e}{Exercice}[]
\newtheorem{p}{Problème}[]
}
%\newtheorem*{e*}{Exercice}
%%% ------------------------------------------------ %%%
%%% Les nouvelles commandes %%%
%\renewcommand{\thede}{\arabic{de}} % Les theorèmes  sont numérotés en chiffres arabes
\def\legendre#1#2{\genfrac(){0.2pt}0{#1}{#2}} % Symbole de legendre
%




\def\restriction#1#2{\mathchoice %Notation restriction
              {\setbox1\hbox{${\displaystyle #1}_{\scriptstyle #2}$}
              \restrictionaux{#1}{#2}}
              {\setbox1\hbox{${\textstyle #1}_{\scriptstyle #2}$}
              \restrictionaux{#1}{#2}}
              {\setbox1\hbox{${\scriptstyle #1}_{\scriptscriptstyle #2}$}
              \restrictionaux{#1}{#2}}
              {\setbox1\hbox{${\scriptscriptstyle #1}_{\scriptscriptstyle #2}$}
              \restrictionaux{#1}{#2}}}
\def\restrictionaux#1#2{{#1\,\smash{\vrule height .8\ht1 depth .85\dp1}}_{\,#2}} 
\newcommand{\bigslant}[2]{{\raisebox{.1em}{$#1$}\left/\raisebox{-.2em}{$#2$}\right.}} % Symbole quotient oblique exemple \bigslant{\Z}{n\Z}
\newcommand{\calG}{\mathcal{G}} % Les symboles de maths 
\newcommand{\calA}{\mathcal{A}}
\newcommand{\calU}{\mathcal{U}}
\newcommand{\calS}{\mathcal{S}}
\newcommand{\calF}{\mathcal{F}}
\newcommand{\calC}{\mathcal{C}}
\newcommand{\calB}{\mathcal{B}}
\newcommand{\calD}{\mathcal{D}}
\newcommand{\calE}{\mathcal{E}}
\newcommand{\calM}{\mathcal{M}}
\newcommand{\calN}{\mathcal{N}}
\newcommand{\calH}{\mathcal{H}}
\newcommand{\calI}{\mathcal{I}}
\newcommand{\calL}{\mathcal{L}}
\newcommand{\calR}{\mathcal{R}}
\newcommand{\calP}{\mathcal{P}}
\newcommand{\calO}{\mathcal{O}}
\newcommand{\calT}{\mathcal{T}}
\newcommand{\calV}{\mathcal{V}}
\newcommand{\calZ}{\mathcal{Z}}
\newcommand{\EE}{\mathbb{E}}
\newcommand{\RR}{\mathbb{R}}
\newcommand{\PP}{\mathbb{P}}
\newcommand{\jP}{\mathbb{P}}
\newcommand{\KK}{\mathbb{K}}
\newcommand{\CC}{\mathbb{C}}
\newcommand{\ZZ}{\mathbb{Z}}
\newcommand{\NN}{\mathbb{N}}
\newcommand{\UU}{\mathbb{U}}
\newcommand{\QQ}{\mathbb{Q}}
\newcommand{\FF}{\mathbb{F}}
\newcommand{\bbP}{\mathbb{P}}
\newcommand{\OFP}{(\Omega,\calF,\bbP)}
\newcommand{\Pa}{\mathscr{P}}
\newcommand{\Oa}{\mathscr{O}}
\newcommand{\Ta}{\mathscr{T}}
\newcommand{\Sa}{\mathscr{S}}
\newcommand{\car}{\mathds{1}} %Les symboles de maths 
%%% --------------------------------------- %%%


\usetikzlibrary{matrix,positioning}


\newcommand{\titre}[1]{%
	\vspace*{\fill}  % Espace flexible avant
	\begin{center}
		\begin{tikzpicture}
			\matrix [matrix of nodes]{
				|[draw=gold2,
				very thick,
				font=\fontsize{40}{48}\selectfont,
				rounded corners=10pt,
				inner sep=0.4cm,
				fill=backg!80,
				text=gold]|
				{\textsc{\ #1\ }}  \\
			} ;
		\end{tikzpicture}
		
		\vspace{2cm}  % Espace entre le titre et votre nom
		
		{\Huge\color{gold2}\textsc{Louis-Thibault Gauthier}}  % Votre nom en plus petit et assorti
		
		\vspace{1cm}  % Espace entre votre nom et la date
		
		{\large\color{gold2}\textsc{\today}}  % Date de production du document
	\end{center}
	\vspace*{\fill}  % Espace flexible après
}



\fancypagestyle{plain}{
	\fancyhf{} %Clear Everything.
	\renewcommand\headrulewidth{1pt}
	%\fancyfoot[C]{\thepage} %Page Number
	\renewcommand{\headrule}{\color{gold2}\hrule width\headwidth height\headrulewidth \vskip-\headrulewidth} %0pt for no rule, 2pt thicker etc...
	\renewcommand{\footrule}{\color{gold2}\hrule width\headwidth height\headrulewidth \vskip-\headrulewidth} %0pt for no rule, 2pt thicker etc...
	\fancyfoot[R]{Session 2027}
	\fancyfoot[L]{Louis-Thibault Gauthier}
	%\fancyfoot[C]{Page \thepage/\pageref{LastPage}}
	\fancyfoot[C]{Page \number\numexpr\thepage+1\relax/\number\numexpr\getpagerefnumber{LastPage}+1\relax}
	%	\fancyhead[L]{TOP LEFT, EVEN PAGES}
	%	\fancyhead[R]{\small \textcolor{gold}{Louis-Thibault GAUTHIER}}
	%	\fancyhead[L]{{\small \textcolor{gold}{OZANAM}}	
	\fancyhead[L]{\textcolor{gold2}{Préparation aux oraux de l'agrégation externe}}
	\fancyhead[C]{\textcolor{gold2}{Mathématiques}}
	\fancyhead[R]{\textcolor{gold2}{Concours standard}}
}
	
	
	\pagestyle{plain}


% Ajouter à la toute fin du préambule :
\usepackage{hyperref}
\hypersetup{
	colorlinks=true,
	linkcolor=gold2,
	urlcolor=bleufoncé,
	citecolor=bleufoncé,
	filecolor=bleufoncé,
	linktoc=all
}

% Style de la table des matières
\renewcommand{\contentsname}{\color{gold2}Table des matières}

