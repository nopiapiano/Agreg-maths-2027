\documentclass[a4paper,10pt,oneside,twocolumn,landscape]{book}


\pagenumbering{arabic}



%%% Le chargement des packages %%%


\usepackage{amssymb}
\setlength{\columnsep}{1cm}
\usepackage{multicol}



\usepackage{eurosym}
\usepackage{graphics}
\usepackage{amsmath}
\usepackage{amsthm}  % DOIT être chargé AVANT de définir les théorèmes
% For Landau's notation
\usepackage{mismath}

\usepackage{stmaryrd} % Where \llbracket and \rrbracket are defined

\usepackage[french]{babel}
\usepackage[utf8]{inputenc}  
% \usepackage[latin1]{inputenc}
\usepackage[T1]{fontenc}  
\usepackage{pstricks-add}
\usepackage{graphicx}
\usepackage[T1]{fontenc}
\usepackage{babel,varioref,float} % polices
\usepackage{pgf,tikz} % permet de créer comme pstricks des figures en code LateX
\let\clipbox\relax

\usetikzlibrary{calc,decorations.pathmorphing,arrows}


\usepackage{pgflibraryarrows} % librairie liée à tikz
\usepackage{pgflibrarysnakes}
\usepackage{xcolor} % module de couleur pour tikz
\usepackage{pst-eucl}
\usepackage{titlesec} % Personnaliser les titres
\usepackage{graphicx}


%\theoremstyle{plain}
\newtheorem*{thm}{Théorème}
\newtheorem*{de}{Définition}
\newtheorem*{lem}{Lemme}


\usepackage{amsmath,amsfonts,graphicx,geometry}
\geometry{top=1.9cm,left=1.2cm,right=1.2cm,bottom=1.6cm}
\usepackage{multicol}



\usepackage{enumitem}

\usepackage{calc,color}
\usepackage{fancybox}
\usepackage{lastpage}
\usepackage{fancyhdr}
\usepackage{lipsum}

\usepackage{marginnote}
\usepackage{fontawesome}

\usepackage[weather]{ifsym}
\usepackage{pifont}


%\usepackage[utf8]{inputenc}    % Pour utiliser les lettres accentuées
%\usepackage[francais]{babel}    % Pour la langue française
\usepackage{enumitem} % Itemize, enumerate etc.
%\usepackage{enumerate} % Idem, mais a ne pas activer ensemble
\usepackage{amscd,amsmath} % Les packages de Maths
\usepackage{color} % les images
\usepackage{index} % Pour pouvoir créer un index
%\usepackage{titlesec} % Personnaliser les titres
\usepackage{mathdots} % Faire des petits points verticaux, en oblique etc
\usepackage{import} % Pouvoir importer des fichers dans un autre dossier
\usepackage[all]{xy} % Faire des zoulis diagrammes
 \usepackage[cyr]{aeguill} % Guillemets en francais et autres
\usepackage{latexsym,stmaryrd,xypic,mathrsfs} % Différents package pour faire des maths
%%% ------------------------------------------ %%%
\usepackage{pgf,tikz} % permet de créer comme pstricks des figures en code LateX



\usepackage[most]{tcolorbox}
\usepackage{adjustbox}

%\usepackage{lmodern} % Police moderne
\usepackage{xcolor}


% Style du bloc (SANS ombre)
\tcbset{
  developpementsbox/.style={
    enhanced,
    sharp corners=south,  % coins carrés en bas
    colback=white,
    colframe=gold,
   % boxrule=1.2pt,
    borderline west={3mm}{0pt}{gold}, % barre épaisse à gauche
    left=10pt,
    right=8pt,
    top=6pt,
    bottom=6pt,
    before skip=12pt,
    after skip=12pt,
  }
}

% Style des items : numérotation classique
\newlist{devlist}{enumerate}{1}
\setlist[devlist,1]{%
  label=\textbf{\arabic*.},
  left=0pt, labelsep=1em, itemsep=0.7em,
  font=\normalsize
}

% Environnement avec le titre toujours affiché
\newenvironment{developpements}{%
  \begin{tcolorbox}[developpementsbox]
  \textbf{\large Liste des développements :}
  \vspace{4pt}
  \begin{devlist}
}{%
  \end{devlist}
  \end{tcolorbox}
}






\newcommand{\poubelle}[1]{} % Commande Poubelle : "\poubelle{ blabla}" et le blabla sera ignoré, même si il tient en 200 lignes





\definecolor{albertBlue1}{RGB}{34, 20, 81}
\definecolor{albertBlue2}{RGB}{54, 37, 157}
\definecolor{albertBlue3}{RGB}{173, 218, 247}

\definecolor{inter}{RGB}{156, 66, 167}

%%% Les couleurs %%%
\definecolor{fondtitre}{rgb}{0.20,0.43,0.09}  % vert fonce
\definecolor{gold}{RGB}{223,179,85}  % Gold
%\definecolor{gold2}{RGB}{255,214,100}  % Gold
\definecolor{gold2}{RGB}{53,93,173}   % Gold2 %bleu
\definecolor{gold3}{RGB}{255,214,100} 
\definecolor{goldinv}{RGB}{32,76,170}
\definecolor{bleu}{RGB}{53,93,173}  
\definecolor{vert}{RGB}{155,205,50}  
\definecolor{vertfoncé}{RGB}{0,128,32}  
\definecolor{bleufoncé}{RGB}{0,32,128}  
\definecolor{grenat}{RGB}{110,11,20}  
\definecolor{amande}{RGB}{130,196,108} 
\definecolor{magentapink}{RGB}{204,51,139}   
\definecolor{glauque}{RGB}{100,155,136}   
\definecolor{oeuf}{RGB}{239,213,178}   
\definecolor{bisque}{RGB}{255,228,196}   
\definecolor{inter}{RGB}{156, 66, 167}
%\definecolor{backg}{RGB}{185,220,200}   
%\definecolor{backg}{RGB}{215,224,240}
\definecolor{crème}{RGB}{253,247,196}  
\definecolor{noirleger}{RGB}{79,77,81}  
\definecolor{marron}{RGB}{88,41,0} 
\definecolor{gris}{RGB}{150,150,150} 
\definecolor{rougem}{RGB}{200,0,0} 
\definecolor{vertm}{RGB}{0,200,0} 
\definecolor{bleum}{RGB}{0,0,200} 
\definecolor{backg}{RGB}{252, 233, 240}





\setcounter{secnumdepth}{1} %  Profondeur des Chapitre/Sections/Sous-sections Etc

\setcounter{tocdepth}{2}  % Profondeur de la table principale



\titleformat{\chapter}[display]
{\clearpage\normalfont\centering\color{gold3}}
{\vspace*{\fill}\vspace{-5em}\fontsize{50}{48}\selectfont\bfseries\chaptertitlename\space\thechapter}
{20pt}
{\fontsize{40}{48}\selectfont\bfseries\thispagestyle{empty}}
[\vspace*{\fill}]
\titlespacing*{\chapter}
{0pt}     
{0pt}      
{0pt}




%\titleformat{\section}
%{\normalfont\Large\bfseries\color{gold2}}  % Format du titre
%{\thesection}   % Le numéro de section
%{1em}          % Espace entre le numéro et le titre
%{}     


%\makeatletter 
%\renewcommand\thesection{\@arabic\c@section} 
%\makeatothe

%\usepackage[export]{adjustbox}


%\renewcommand{\thechapter}{\Roman{chapter}}





% Properties
\definecolor{propColor}{RGB}{0, 255, 0}
\newcommand{\Props}[2][]{
	\begin{tcolorbox}[colback=propColor!5!white,colframe=propColor!75!black,title=Propositions #1]
		#2
\end{tcolorbox}}
\newcommand{\Prop}[2][]{
	\begin{tcolorbox}[colback=vertfoncé!5!white,colframe=vertfoncé!75!black,title=Proposition #1]
		#2
\end{tcolorbox}}

% Definitions
\definecolor{defColor}{RGB}{255, 255, 255}
\newcommand{\Defs}[2][]{
	\begin{tcolorbox}[colback=defColor!5!white,colframe=defColor!75!black,title=Definitions #1]
		#2
\end{tcolorbox}}
\newcommand{\Def}[2][]{
	\begin{tcolorbox}[colback=defColor!5!white,colframe=defColor!75!black,title=Definition #1]
		#2
\end{tcolorbox}}

% Theorems
\definecolor{theoremColor}{RGB}{255, 0, 0}
\newcommand{\Thm}[2][]{
	\begin{tcolorbox}[colback=bleufoncé!5!white,colframe=bleufoncé!75!black,title=Théorème #1]
		#2
\end{tcolorbox}}

% Applications
\newcommand{\Ap}[2][]{
	\begin{tcolorbox}[colback=grenat!5!white,colframe=grenat!75!black,title=Application #1]
		#2
\end{tcolorbox}}

% Examples
\newcommand{\example}[0]{
	\vspace{1em}
	{\bf Example :}
}

% Diverse things to highlight
\newcommand{\highlight}[1]{
	\vspace{1em}
	{\bf #1}
	\vspace{0.25em}
	
}



%\newtheorem{prop}{Proposition}
%\newtheorem{lem}{Lemme}
%\newtheorem{cor}{Corollaire}

\poubelle{
\newtheorem{thm}{Théorème}[]
\newtheorem{de}{Définition}[]
\newtheorem{prop}{Proposition}[]
\newtheorem{ap}{Application}[]
\newtheorem{rem}{Remarque}[]
\newtheorem{cex}{Contre-Exemple}[]
\newtheorem{cor}{Corollaire}[]
\newtheorem{cocor}{Cocorollaire}[]
\newtheorem{ex}{Exemple}[]
\newtheorem{lem}{Lemme}[]
\newtheorem{de-prop}{Définition-Proposition}[]
%\theoremstyle{definition} % à partir de là, les newtheorem seront en lettre droite contrairement à italique.
\newtheorem{etape}{Etape}[]
\newtheorem{e}{Exercice}[]
\newtheorem{p}{Problème}[]
}
%\newtheorem*{e*}{Exercice}
%%% ------------------------------------------------ %%%
%%% Les nouvelles commandes %%%
%\renewcommand{\thede}{\arabic{de}} % Les theorèmes  sont numérotés en chiffres arabes
\def\legendre#1#2{\genfrac(){0.2pt}0{#1}{#2}} % Symbole de legendre
%




\def\restriction#1#2{\mathchoice %Notation restriction
              {\setbox1\hbox{${\displaystyle #1}_{\scriptstyle #2}$}
              \restrictionaux{#1}{#2}}
              {\setbox1\hbox{${\textstyle #1}_{\scriptstyle #2}$}
              \restrictionaux{#1}{#2}}
              {\setbox1\hbox{${\scriptstyle #1}_{\scriptscriptstyle #2}$}
              \restrictionaux{#1}{#2}}
              {\setbox1\hbox{${\scriptscriptstyle #1}_{\scriptscriptstyle #2}$}
              \restrictionaux{#1}{#2}}}
\def\restrictionaux#1#2{{#1\,\smash{\vrule height .8\ht1 depth .85\dp1}}_{\,#2}} 
\newcommand{\bigslant}[2]{{\raisebox{.1em}{$#1$}\left/\raisebox{-.2em}{$#2$}\right.}} % Symbole quotient oblique exemple \bigslant{\Z}{n\Z}
\newcommand{\calG}{\mathcal{G}} % Les symboles de maths 
\newcommand{\calA}{\mathcal{A}}
\newcommand{\calU}{\mathcal{U}}
\newcommand{\calS}{\mathcal{S}}
\newcommand{\calF}{\mathcal{F}}
\newcommand{\calC}{\mathcal{C}}
\newcommand{\calB}{\mathcal{B}}
\newcommand{\calD}{\mathcal{D}}
\newcommand{\calE}{\mathcal{E}}
\newcommand{\calM}{\mathcal{M}}
\newcommand{\calN}{\mathcal{N}}
\newcommand{\calH}{\mathcal{H}}
\newcommand{\calI}{\mathcal{I}}
\newcommand{\calL}{\mathcal{L}}
\newcommand{\calR}{\mathcal{R}}
\newcommand{\calP}{\mathcal{P}}
\newcommand{\calO}{\mathcal{O}}
\newcommand{\calT}{\mathcal{T}}
\newcommand{\calV}{\mathcal{V}}
\newcommand{\calZ}{\mathcal{Z}}
\newcommand{\EE}{\mathbb{E}}
\newcommand{\RR}{\mathbb{R}}
\newcommand{\PP}{\mathbb{P}}
\newcommand{\jP}{\mathbb{P}}
\newcommand{\KK}{\mathbb{K}}
\newcommand{\CC}{\mathbb{C}}
\newcommand{\ZZ}{\mathbb{Z}}
\newcommand{\NN}{\mathbb{N}}
\newcommand{\UU}{\mathbb{U}}
\newcommand{\QQ}{\mathbb{Q}}
\newcommand{\FF}{\mathbb{F}}
\newcommand{\bbP}{\mathbb{P}}
\newcommand{\OFP}{(\Omega,\calF,\bbP)}
\newcommand{\Pa}{\mathscr{P}}
\newcommand{\Oa}{\mathscr{O}}
\newcommand{\Ta}{\mathscr{T}}
\newcommand{\Sa}{\mathscr{S}}
\newcommand{\car}{\mathds{1}} %Les symboles de maths 
%%% --------------------------------------- %%%


\usetikzlibrary{matrix,positioning}


\newcommand{\titre}[1]{%
	\vspace*{\fill}  % Espace flexible avant
	\begin{center}
		\begin{tikzpicture}
			\matrix [matrix of nodes]{
				|[draw=gold2,
				very thick,
				font=\fontsize{40}{48}\selectfont,
				rounded corners=10pt,
				inner sep=0.4cm,
				fill=backg!80,
				text=gold]|
				{\textsc{\ #1\ }}  \\
			} ;
		\end{tikzpicture}
		
		\vspace{2cm}  % Espace entre le titre et votre nom
		
		{\Huge\color{gold2}\textsc{Louis-Thibault Gauthier}}  % Votre nom en plus petit et assorti
		
		\vspace{1cm}  % Espace entre votre nom et la date
		
		{\large\color{gold2}\textsc{\today}}  % Date de production du document
	\end{center}
	\vspace*{\fill}  % Espace flexible après
}



\fancypagestyle{plain}{
	\fancyhf{} %Clear Everything.
	\renewcommand\headrulewidth{1pt}
	%\fancyfoot[C]{\thepage} %Page Number
	\renewcommand{\headrule}{\color{gold2}\hrule width\headwidth height\headrulewidth \vskip-\headrulewidth} %0pt for no rule, 2pt thicker etc...
	\renewcommand{\footrule}{\color{gold2}\hrule width\headwidth height\headrulewidth \vskip-\headrulewidth} %0pt for no rule, 2pt thicker etc...
	\fancyfoot[R]{Session 2027}
	\fancyfoot[L]{Louis-Thibault Gauthier}
	%\fancyfoot[C]{Page \thepage/\pageref{LastPage}}
	\fancyfoot[C]{Page \number\numexpr\thepage+1\relax/\number\numexpr\getpagerefnumber{LastPage}+1\relax}
	%	\fancyhead[L]{TOP LEFT, EVEN PAGES}
	%	\fancyhead[R]{\small \textcolor{gold}{Louis-Thibault GAUTHIER}}
	%	\fancyhead[L]{{\small \textcolor{gold}{OZANAM}}	
	\fancyhead[L]{\textcolor{gold2}{Préparation aux oraux de l'agrégation externe}}
	\fancyhead[C]{\textcolor{gold2}{Mathématiques}}
	\fancyhead[R]{\textcolor{gold2}{Concours standard}}
}
	
	
	\pagestyle{plain}


% Ajouter à la toute fin du préambule :
\usepackage{hyperref}
\hypersetup{
	colorlinks=true,
	linkcolor=gold2,
	urlcolor=bleufoncé,
	citecolor=bleufoncé,
	filecolor=bleufoncé,
	linktoc=all
}

% Style de la table des matières
\renewcommand{\contentsname}{\color{gold2}Table des matières}





\begin{document}
	\onecolumn  % Pour la page de titre
	\begin{titlepage}
		\titre{Oraux - agrégation externe 2027}
	\end{titlepage}
	
	\onecolumn  % Pour le texte d'introduction

\tableofcontents  % Table des matières principale

\thispagestyle{empty}



\chapter{Leçons d'algèbre}


\twocolumn  % Reprise du format deux colonnes après le chapitre




\newpage

\section{101 : Groupe opérant sur un ensemble. Exemples et applications.}


\begin{developpements}
	\item Théorème de la base de Burnside. Un max de maths p13.
\end{developpements}
 

\Prop
{
	Une action est fidèle si et seulement si $\phi$, le morphisme associé est injectif ssi l'intersection de tous les stabilisateurs vaut $e_G$.
}


\begin{proof}
	Si $\phi$ est injectif, soit $g\in  \bigcap_x \textup{Stab(x)}$. Alors $\forall x \in X$, $g\cdot x = x$, et donc par injectivité $g = e_G$. Si $\bigcap_x \textup{Stab(x)} = \{e_G\}$, alors il est clair que $\phi$ est injectif.
\end{proof}


\Thm[(Formule des classes)]
{
	$X$ est fini :
	\[
		\vert X \vert = \sum_{x \in \bigslant{X}{ \sim}} \vert \textup{Orb}(x)\vert
	\]
	Et on a : $\textup{Orb}(x)$ et $\bigslant{G}{\textup{Stab}(x)}$ sont en bijection.
}

\begin{proof}
	Si $G$ agit sur $X$, les orbites sont des classes d'équivalence. Elles sont disjointes, et forment une partition de $X$, reste à évaluer le cardinal d'une classe, or on a une bijection, pour $x \in X$ :
	\[
		f : \begin{array}{cccc} \bigslant{G}{\textup{Stab}(x)} & \to & G x\\ g \textup{Stab}(x) & \mapsto & g\cdot x\end{array}
	\]
	par passage au quotient de $G \to Gx,g \mapsto g \cdot x$.
\end{proof}


\Prop[(Formule de Burnside :)]
{
	Si $G$ et $X$ sont finis, alors
	\[
		\left| \bigslant{X}{\sim} \right| = \frac{1}{\vert G \vert} \sum_{g\in G} \vert \textup{Fix}(g) \vert
	\]
}

\begin{proof}
	Soit $E$ l'ensemble des couples $(g,x)$ où $g\cdot x = x$, alors
	\[
		\vert E \vert = \sum_{g \in G}\vert \textup{Fix}(g)\vert = \sum_{x\in X} \vert \textup{Stab}(x) \vert
	\]
	Si $\Omega$ est une transversale de $X$ (donc de cardinal le nombre des orbites), on a :
	\[ 
		\vert E \vert = \sum_{x \in X} \frac{\vert G\vert}{\vert Gx\vert } = \vert G \vert \sum_{ \omega \in \Omega}  \sum_{ x \in \omega} 	\frac{1}{\vert Gx\vert} = \vert G \vert \sum_{ \omega \in \Omega}  \sum_{ x \in \omega} \frac{1}{\vert \omega\vert}  = \vert G  \vert \vert \Omega \vert 
	\]
\end{proof}


\Prop[]
{
	Soit $G$ un groupe de cardinal $p^\alpha(>1)$, alors le centre de $G$ n'est pas réduit à l'élément neutre.
}

\begin{proof} 
	On considère l'action de $G$ sur lui même par automorphisme intérieur. par la formule des classe on a :
	\[
		\vert G \vert = \vert \calZ(G) \vert + \sum_{x\in A} \frac{\vert G \vert}{\vert \textup{Stab}(x) \vert }
	\]
	Avec $A$ une transversale pour l'ensemble des orbites non réduites à un point. On en déduit que puisque le centre est non vide, qu'il est un multiple de $p$.
\end{proof}


\Ap[]
{
	Il n'existe que 2 groupes d'ordre $p^2$ à isomorphisme près. 
}

\begin{proof} 
	D'après la proposition précédente un tel groupe $G$ a son centre de cardinal $p$ ou $p^2$. Si il est de cardinal $p$ un élément de $G$ est dans $\calZ(G)$ si et seulement si son centralisateur $Z_G(g)$ est $G$. Comme le centralisateur d'un élément $g \in G\setminus \calZ(G)$ contient $g$ et contient $\calZ(G)$, il est donc d'ordre $>p$. Donc $\calZ(G)= G$, ce qui donne une contradiction. Donc $G$ est abélien. Soit $G$ est monogène et à se moment là il est isomorphe à $\bigslant{\ZZ}{p^2\ZZ}$, sinon soit $g \in G$, qui n'est pas l'élément unité, le sous-groupe $H$ engendré par $g$ est d'ordre $p$. Donc $\bigslant{G}{H}$ est d'ordre $p$ donc isomorphe à $\bigslant{\ZZ}{p\ZZ}$. Donc $G$ est isomorphe à $\bigslant{\ZZ}{p\ZZ} \times \bigslant{\ZZ}{p\ZZ}$.
\end{proof}




\section{102 : Groupe des nombres complexes de module $1$. Sous-groupes des racines de l'unité. Applications.}
\begin{itemize}
	\item
\end{itemize}

\section{103 : Conjugaison dans un groupe. Exemples de sous-groupes distingués et de groupes quotients. Applications.}
\begin{itemize}
	\item Théorème de la base de Burnside. Un max de maths p13.
\end{itemize}


\section{104 : Groupes finis. Exemples et applications.}
\begin{itemize}
	\item Théorème de la base de Burnside. Un max de maths p13.
\end{itemize}


\section{105  : Groupe des permutations d’un ensemble fini. Applications.}
\begin{itemize}
	\item 
\end{itemize}


\section{106 : Groupe linéaire d’un espace vectoriel de dimension finie $E$, sous-groupes de $\textup{GL}(E)$. Applications.}
\begin{itemize}
	\item
\end{itemize}


\section{108 :  Exemples de parties génératrices d’un groupe. Applications.}

\begin{itemize}
	\item Théorème de la base de Burnside. Un max de maths p13.
\end{itemize}


\section{120 : Anneaux $\bigslant{\ZZ}{n\ZZ}$. Applications.}

\begin{itemize}
	\item 
\end{itemize}

\section{121 : Nombres premiers. Applications.}

\begin{itemize}
	\item 
\end{itemize}

\section{122 : Anneaux principaux. Applications.}

\begin{itemize}
	\item 
\end{itemize}

\section{123 : Corps finis. Applications.}

\begin{itemize}
	\item
\end{itemize}

\section{125 : Extensions de corps. Exemples et applications.}

\begin{itemize}
	\item 
\end{itemize}

\section{127 : Exemples de nombres remarquables. Exemples d’anneaux de nombres remarquables. Applications.}


\section{141 : Polynômes irréductibles à une indéterminée. Corps de rupture. Exemples et applications.}

\begin{itemize}
	\item 
\end{itemize}

\section{142 : PGCD et PPCM, algorithmes de calcul. Applications.}

\begin{itemize}
	\item
\end{itemize}

\section{144 : Racines d’un polynôme. Fonctions symétriques élémentaires. Exemples et applications.}

\begin{itemize}
	\item
\end{itemize}

\section{148 : Dimension d’un espace vectoriel (on se limitera au cas de la dimension finie). Rang. Exemples et applications.}

\begin{itemize}
	\item
\end{itemize}


\section{149 : Déterminant. Exemples et applications.}

\begin{itemize}
	\item 
\end{itemize}


\section{150 : Polynômes d’endomorphisme en dimension finie. Réduction d’un endomorphisme en dimension finie. Applications.}

\begin{itemize}
	\item 
\end{itemize}

\section{151 : Sous-espaces stables par un endomorphisme ou une famille d’endomorphismes d’un espace vectoriel de dimension finie. Applications.}

\begin{itemize}
	\item 
\end{itemize}

\section{152 : Endomorphismes diagonalisables en dimension finie.}

\begin{itemize}
	\item 
\end{itemize}

\section{153 : Valeurs propres, vecteurs propres. Calculs exacts ou approchés d’éléments
	propres. Applications.}

\begin{itemize}
	\item 
\end{itemize}

\section{155 : Exponentielle de matrices. Applications.}

\begin{itemize}
	\item 
\end{itemize}




\section{156 : Endomorphismes trigonalisables. Endomorphismes nilpotents.}

\begin{itemize}
	\item 
\end{itemize}

\section{157 : Matrices symétriques réelles, matrices hermitiennes.}

\begin{itemize}
	\item 
\end{itemize}

\section{158 : Endomorphismes remarquables d’un espace vectoriel euclidien (de dimension
	finie).}

\begin{itemize}
	\item 
\end{itemize}


\section{159 : Formes linéaires et dualité en dimension finie. Exemples et applications.}

\begin{itemize}
	\item 
\end{itemize}


\section{161 : Espaces vectoriels et espaces affines euclidiens : distances, isométries.}

\begin{itemize}
	\item 
\end{itemize}

\section{162 : Systèmes d’équations linéaires ; opérations élémentaires, aspects algorithmiques et conséquences théoriques.}

\begin{itemize}
	\item 
\end{itemize}

\section{170 : Formes quadratiques sur un espace vectoriel de dimension finie. Orthogonalité, Applications.}

\begin{itemize}
	\item 
\end{itemize}

\section{171 : Formes quadratiques réelles. Coniques. Exemples et applications.}

\begin{itemize}
	\item 
\end{itemize}


\section{181 : Convexité dans $\RR^n$. Applications en algèbre et en géométrie.}

\begin{itemize}
	\item 
\end{itemize}


\section{190 : Méthodes combinatoires, problèmes de dénombrement.}

\begin{itemize}
	\item 
\end{itemize}

\section{191 : Exemples d’utilisation des techniques d’algèbre en géométrie.}

\begin{itemize}
	\item 
\end{itemize}





\onecolumn

\chapter{Leçons d'analyse}

\twocolumn



\section{201 : Espaces de fonctions ; exemples et applications.}

\begin{itemize} 
	\item Théorèmes de Banach-Alaoglu. 40 dev d'analyse p27.
\end{itemize}

\section{203 : Utilisation de la notion de compacité.}

\begin{itemize}
	\item 
\end{itemize}

\section{204 : Connexité. Exemples et applications.}

\begin{itemize}
	\item 
\end{itemize}

\section{205 : Espaces complets. Exemples et applications.}

\begin{itemize}
	\item Théorèmes de Banach-Alaoglu. 40 dev d'analyse p27.
\end{itemize}



\section{206 : Connexité. Exemples et applications.}

\begin{itemize}
	\item Exemples d’utilisation de la notion de dimension finie en analyse.
\end{itemize}


\section{208 : Espaces vectoriels normés, applications linéaires continues. Exemples.}

\begin{itemize}
	\item Théorèmes de Banach-Alaoglu. 40 dev d'analyse p27.
\end{itemize}

\section{209 : Approximation d’une fonction par des fonctions régulières. Exemples d’applications.}

\begin{itemize}
	\item 
\end{itemize}

\section{213 : Espaces de Hilbert. Exemples d’applications.}

\begin{itemize}
	\item Théorèmes de Banach-Alaoglu. 40 dev d'analyse p27.
\end{itemize}

\section{214 : Théorème d’inversion locale, théorème des fonctions implicites. Illustrations en analyse et en géométrie.}

\begin{itemize}
	\item 
\end{itemize}

\section{215 : Applications différentiables définies sur un ouvert de $\RR^n$. Exemples et applications.}

\begin{itemize}
	\item 
\end{itemize}

\section{218 : Formules de Taylor. Exemples et applications.}

\begin{itemize}
	\item 
\end{itemize}

\section{219 : Extremums : existence, caractérisation, recherche. Exemples et applications.}

\begin{itemize}
	\item 
\end{itemize}

\section{220 : Illustrer par des exemples la théorie des équations différentielles ordinaires.}

\begin{itemize}
	\item 
\end{itemize}

\section{221 : Équations différentielles linéaires. Systèmes d’équations différentielles linéaires. Exemples et applications.}

\begin{itemize}
	\item 
\end{itemize}


\section{223 : Suites numériques. Convergence, valeurs d’adhérence. Exemples et applications.}

\begin{itemize}
\item 
\end{itemize}


\section{224 : Exemples de développements asymptotiques de suites et de fonctions.}

\begin{itemize}
	\item 
\end{itemize}

\section{226 : Suites vectorielles et réelles définies par une relation de récurrence $u_{n+1} = f(u_n)$. Exemples. Applications à la résolution approchée d’équations.}

\begin{itemize}
	\item 
\end{itemize}

\section{228 : Continuité et dérivabilité des fonctions réelles d’une variable réelle. Exemples et applications.}

\begin{itemize}
	\item 
\end{itemize}

\section{229 : Fonctions monotones. Fonctions convexes. Exemples et applications.}

\begin{itemize}
	\item 
\end{itemize}

\section{230 : Séries de nombres réels ou complexes. Comportement des restes ou des sommes partielles des séries numériques. Exemples.}

\begin{itemize}
	\item 
\end{itemize}

\section{234 : Fonctions et espaces de fonctions Lebesgue-intégrables.}

\begin{itemize}
	\item 
\end{itemize}

\section{235 : Problèmes d’interversion de symboles en analyse.}

\begin{itemize}
	\item 
\end{itemize}

\section{236 : Illustrer par des exemples quelques méthodes de calcul d’intégrales de fonctions d’une ou plusieurs variables.}

\begin{itemize}
	\item 
\end{itemize}

\section{239 : Fonctions définies par une intégrale dépendant d’un paramètre. Exemples et applications.}

\begin{itemize}
	\item 
\end{itemize}

\section{241 : Suites et séries de fonctions. Exemples et contre-exemples.}

\begin{itemize}
	\item 
\end{itemize}

\section{243 : Séries entières, propriétés de la somme. Exemples et applications.}

\begin{itemize}
	\item 
\end{itemize}

\section{245 : Fonctions holomorphes et méromorphes sur un ouvert de $\CC$. Exemples et applications.}

\begin{itemize}
	\item 
\end{itemize}

\section{246 : Séries de Fourier. Exemples et applications.}

\begin{itemize}
	\item 
\end{itemize}

\section{250 : Transformation de Fourier. Applications.}

\begin{itemize}
	\item 
\end{itemize}

\section{253 : Utilisation de la notion de convexité en analyse.}

\begin{itemize}
	\item 
\end{itemize}


\section{261 : Loi d’une variable aléatoire: caractérisations, exemples, applications}

\begin{itemize}
	\item 
\end{itemize}

\section{262 : Convergences d’une suite de variables aléatoires. Théorèmes limite. Exemples et applications}

\begin{itemize}
	\item 
\end{itemize}


\section{264 : Variables aléatoires discrètes. Exemples et applications.}

\begin{itemize}
	\item 
\end{itemize}


\section{266 : Utilisation de la notion d’indépendance en probabilités}

\begin{itemize}
\item 
\end{itemize}




\setcounter{secnumdepth}{1} %  Profondeur des Chapitre/Sections/Sous-sections Etc



\onecolumn

\chapter{Développement d'algèbre}

\twocolumn




\section{Théorème de la base de Burnside}
	 

Soit $G$ un $p$-groupe (groupe fini d'ordre une puissance de $p$ premier).



\begin{de}
	\begin{itemize}
		\item Un sous-groupe maximal de $G$ est un sous-groupe strict de $G$ et maximal pour l'inclusion. On note $\calM$ leur ensemble. 
		\item Le normalisateur de $H$ dans $G$, $N_G(H)$ est le sous-groupe de $G$ qui laisse stable $H$ par l'action de conjugaison.
	\end{itemize}
\end{de}


\begin{lem}
	Soit $H \in \calM$, alors $H \triangleleft G$, et $\bigslant{G}{H} = \bigslant{\ZZ}{p\ZZ}$.
\end{lem}
\begin{proof}
	On fait agir $H$ sur $\bigslant{G}{H}$ par multiplication des classes à gauche, on a, par la formule des classes 
	\begin{equation}\label{100117}
		0 \equiv \vert \bigslant{G}{H} \vert \equiv  \left\vert \left( \bigslant{G}{H} \right)^H \right\vert [p]
	\end{equation}
	\noindent Donc  $p$ divise le cardinal de $\left( \bigslant{G}{H} \right)^H $. Or :
	\[
		\begin{array}{lll}
			gH \in \left( \bigslant{G}{H} \right)^H  &\Longleftrightarrow& \forall h \in H;  hg H = gH\\
			 & \Longleftrightarrow&  Hg H = gH \\
			 &\Longleftrightarrow&  Hg = gH\\
			  &\Longleftrightarrow& g \in N_G(H)
		\end{array}
	\]
	On peut alors considérer $\psi : \left( \begin{array}{cccc} N_G(H) & \to &\left( \bigslant{G}{H} \right)^H \\ g & \mapsto &gH \end{array} \right)$ application qui est donc surjective, dont le nombre d'antécédents d'un élément est $\vert H \vert$, donc on a l'égalité $\vert N_G(H) \vert = \vert H \vert \times  \underbrace{\left\vert \left(\bigslant{G}{H} \right)^H  \right\vert}_{\underset{ \mbox{\tiny{par} } (\ref{100117}) }{ \geq p}}$, et donc on a  $\vert N_G(H) \vert > \vert H \vert$, et donc $H \subsetneq N_G(H) \subseteq G$, ainsi par maximalité :
	\[ 
		N_G(H) = G  
	\]
	C'est à dire $H \triangleleft G$.
	\\
	De plus, comme $H$ est maximal dans $G$, $\bigslant{G}{H}$ n'a pas de sous-groupe propre (correspondance des sous-groupes de $\bigslant{G}{H}$) donc $\bigslant{G}{H}$ est cyclique ( et de cardinal une puissance de $p$) ce qui entraine :  
	\[ 
		\bigslant{G}{H} = \bigslant{\ZZ}{p\ZZ} 
	\]
\end{proof}


\begin{thm}
	Les parties génératrices minimales de $G$ ont le même cardinal.
\end{thm}
\begin{proof}
	Considérons le sous groupe $\Phi(G) := \bigcap_{H\in \calM} H \triangleleft G$, notons $\pi : G \to \bigslant{G}{\Phi(G)}$. \\
	Soit $ H\in \calM$, d'après le lemme précédent, $\bigslant{G}{H}$ est abélien, donc $D(G) \subseteq H$, donc $D(G) \subset \Phi(G)$, donc $\bigslant{G}{\Phi(G)}$ est abélien, en particulier c'est un $\Z$-module. \\
	Soit $x\in G$, soit $H\in \calM$, on note $\sigma : G \to \bigslant{G}{H} = \bigslant{\ZZ}{p\ZZ}$, alors $\sigma(x^p) = p \sigma(x) = 0$, ainsi $x^p \in \textup{Ker}(\sigma) = H$, donc
	\[ 
		\forall x \in G; x^p \in \Phi(G)
	\]
	Ainsi, pour tout $x \in G$ $\pi(x)^p = 1$, ainsi, de la structure de $\ZZ$-module sur $\bigslant{G}{\Phi(G)}$ on en déduit une structure de $\FF_p$-espace vectoriel de dimension finie, dont toutes les familles génératrices minimales sont des bases, et en particulier ont le même cardinal.
\end{proof}

On a démontré que les parties génératrices minimal générant le groupe entier ont même cardinal, seulement pour $\bigslant{G}{\Phi(G)}$, le lemme suivant conclut la preuve :

\begin{lem}
	$(g_i)_{i\in I}$ est génératrice de $G$ si et seulement si $(\pi(g_i))_{i \in I}$ est génératrice de $\bigslant{G}{\Phi(G)}$.
\end{lem}
\begin{proof}
	L'implication directe est immédiate par surjectivité de $\pi$. Pour la réciproque, raisonnons par contraposée. Si $(g_i)$ n'engendre pas $G$, considérons un sous-groupe maximal $H$ de $G$ contenant le sous-groupe engendré par la famille $(g_i)$. Alors $\Phi(G) \subseteq H \subsetneq G$, donc $\pi(H) \subsetneq \bigslant{G}{\Phi(G)}$, et la famille $(\pi(g_i))$ n'engendre pas $\bigslant{G}{\Phi(G)}$.
\end{proof}







\end{document}